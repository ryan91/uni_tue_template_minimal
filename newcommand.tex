%+------------------------------------+
%|       self-defined commands        |
%+------------------------------------+

%            Miscellanious
% -------------------------------------

% over- and underset
% @param 1: What's in the middle
% @param 2: What's above
% @param 3: What's below
\newcommand
{\ouset}[3]
{\overset{#2}{\underset{#3}{#1}}}

% small math text
% Use this if you want to write small text in math mode but do not want it
% to be in italics
% @param 1: The small text
\newcommand
{\smt}[1]
{\mbox{\scriptsize #1}}

% identity
\newcommand
{\id}
{\mbox{id}}

% Rang
\newcommand
{\rg}
{\mbox{rg}}

% German quotation marks
% @param 1: Contents between quotation marks
\newcommand
{\gqm}[1]
{\glqq #1\grqq}

%            Mathematic
% -------------------------------------

    % 1 General Constructs

		% 1.1 Miscellanious / Stuff that turned out to be useful

            % Standard limes to infinity
            %@param 1: Variable
            \renewcommand
            {\liminf}[1]
            {\limlim{#1\rightarrow\infty}}

            % Standart sum to infinity
            \newcommand
            {\suminf}[1]
            {\limsum{#1}{\infty}}

		% 1.2 Left- Right- Bracket- Constructs

			% round bracket
			\newcommand
			{\lrr}[1]
			{\left(#1\right)}

			% angled bracket
			\newcommand
			{\lra}[1]
			{\left[#1\right]}

			% cambered bracket
			\newcommand
			{\lrc}[1]
			{\left\{#1\right\}}

			% angled brackets
			\newcommand{\lrg}[1]
			{\left\langle #1\right\rangle}

			% absolute value
			\newcommand
			{\lrabs}[1]
			{\left|#1\right|}

			% floor
			\newcommand
			{\floor}[1]
			{\left\lfloor#1\right\rfloor}

			% ceiling
			\newcommand
			{\ceiling}[1]
			{\left\lceil#1\right\rceil}

			% vector
			\newcommand
			{\lrv}[1]
			{\lrr{\begin{smallmatrix}#1\end{smallmatrix}}}

		% 1.3 Constructs containing "\limits" command

			% Sum with limits
			% @param 1: Index below the sum
			% @param 2: index above the sum
			\newcommand
			{\limsum}[2]
			{\sum\limits_{#1}^{#2}}

			% Product with limits
			% @param 1: Index below the product
			% @param 2: index above the proudct
			\newcommand
			{\limprod}[2]
			{\prod\limits_{#1}^{#2}}

			% Integral witch limits
			% @param 1: Index below the integral
			% @param 2: index above the integral
			\newcommand
			{\limint}[2]
			{\displaystyle\int\limits_{#1}^{#2}}

			% Limes with limits
			%@param 1: What's under the limes
			\newcommand
			{\limlim}[1]
			{\lim\limits_{#1}}

		% 1.4 Other

			% große Tilde über ein Zeichen
			\newcommand
			{\bigtilde}[1]
			{\overset{\sim}{#1}}

	% 2 Logische Junktoren

		% und- Junktor
		\renewcommand
		{\land}
		{\wedge}

		% oder- Junktor
		\renewcommand
		{\lor}
		{\vee}

		% nicht- Junktor
		\renewcommand
		{\lnot}
		{\neg}

		% exclusive- Or (XOR)- Junktor
		\newcommand
		{\lxor}
		{\oplus}

	% 3 Mengenlehre

		% 3.1 Spezielle Mengen

			% natürliche Zahlen
			\newcommand
			{\mn}
			{\mathbb{N}}

			% reelle Zahlen
			\newcommand
			{\mr}
			{\mathbb{R}}

			% ganze Zahlen
			\newcommand
			{\mz}
			{\mathbb{Z}}

			% rationale Zahlen
			\newcommand
			{\mq}
			{\mathbb{Q}}

			% komplexe Zahlen
			\newcommand
			{\mc}
			{\mathbb{C}}

			% Lösungsmenge
			\newcommand
			{\ml}
			{\mathbb{L}}

			% leere Menge
			\newcommand
			{\mvoid}
			{\emptyset}

			% Potenzmenge
			\newcommand
			{\mpz}
			{\mathcal{P}}

		% 3.2 Verknüfungsglieder von Mengen

			% Vereinigung
			\newcommand
			{\mor}
			{\cup}

			% Schnitt
			\newcommand
			{\mand}
			{\cap}

			% symmetrische Differenz
			\newcommand
			{\mnand}
			{\bigtriangleup}

			% Differenzmenge
			\newcommand
			{\mnot}
			{\setminus}

			% Komplement
			\newcommand
			{\mcp}
			{^\complement}

			% Teilmenge ("part of")
			\newcommand
			{\mpo}
			{\subseteq}

			% "Summe" aller Vereinigungen
			% @param 1: Was unter dem Operator steht
			% @param 2: Was über dem Operator steht
			\newcommand
			{\morsum}[2]
			{\bigcup\limits_{#1}^{#2}}

			% "Summe" aller Schnitte
			% @param 1: Was unter dem Operator steht
			% @param 2: Was über dem Operator steht
			\newcommand
			{\mandsum}[2]
			{\bigcap\limits_{#1}^{#2}}

			% Karthesisches Produkt
			\newcommand
			{\mkth}[2]
			{\prod\limits_{#1}^{#2}}

    % 4 Analysis

        % Grad einer Funktion
        \newcommand
        {\grad}
        {\mathrm{grad}}

        % Umkehrfunktion
        \newcommand
        {\ifu}[1]
        {#1^{-1}}

        % Diff.quotient
        \renewcommand
        {\dq}[1]
        {\frac{\mbox{d}}{\mbox{d}#1}}

        % differenzierbar
        \newcommand
        {\diffbar}
        {differenzierbar\;}

        %dx
        \newcommand
        {\intd}[1]
        {\;\mathrm{d}#1}

	% 5  Lineare Algebra

		% R-Vektorraum
		\newcommand
		{\rvecroom}
		{$\mr$-Vektorraum}

		% linear anhängig
		\newcommand
		{\la}
		{linear abhängig }

		% linear unabhängig
		\newcommand
		{\lu}
		{linear unabhängig }

		% lineares Gleichungssystem
		\newcommand
		{\lgs}
		{lineares Gleichungssystem }

		% Größter gemeinsamer Teiler
		\newcommand
		{\ggT}
		{\mbox{ggT}}
